\documentclass[%singlesided,
               doublesided,
               paper=a4,
               fontsize=11pt
              ]{my-resume}


%%%%%%%%%%%%%%%%%%%%%%%%%%%%%%%%%%%%%%%%%%%%%%%%%%%%%%%%%%%%%%%%%%%%%%%%%%%%%%%%
% set geometry
%%%%%%%%%%%%%%%%%%%%%%%%%%%%%%%%%%%%%%%%%%%%%%%%%%%%%%%%%%%%%%%%%%%%%%%%%%%%%%%%

\setlength\highlightwidth{8cm}
\setlength\headerheight{4cm}            % note that margintop gets added to this value, i.e. the header bar is 5cm
\setlength\marginleft{1cm}
\setlength\marginright{\marginleft}      % needs to be 1.5 times to be actually equal. why?
\setlength\margintop{1cm}
\setlength\marginbottom{1cm}


%%%%%%%%%%%%%%%%%%%%%%%%%%%%%%%%%%%%%%%%%%%%%%%%%%%%%%%%%%%%%%%%%%%%%%%%%%%%%%%%
% FONTS
%%%%%%%%%%%%%%%%%%%%%%%%%%%%%%%%%%%%%%%%%%%%%%%%%%%%%%%%%%%%%%%%%%%%%%%%%%%%%%%%

\RequirePackage{fontspec}
\setmainfont{Carlito}


%%%%%%%%%%%%%%%%%%%%%%%%%%%%%%%%%%%%%%%%%%%%%%%%%%%%%%%%%%%%%%%%%%%%%%%%%%%%%%%%
% COLORS
%%%%%%%%%%%%%%%%%%%%%%%%%%%%%%%%%%%%%%%%%%%%%%%%%%%%%%%%%%%%%%%%%%%%%%%%%%%%%%%%

\colorlet{highlightbarcolor}{blue}
\colorlet{headerbarcolor}{cyan}

\colorlet{headerfontcolor}{white}
\colorlet{accent}{awesome-red}
\colorlet{heading}{black}
\colorlet{emphasis}{black}
\colorlet{body}{black}


%%%%%%%%%%%%%%%%%%%%%%%%%%%%%%%%%%%%%%%%%%%%%%%%%%%%%%%%%%%%%%%%%%%%%%%%%%%%%%%%
% set document
%%%%%%%%%%%%%%%%%%%%%%%%%%%%%%%%%%%%%%%%%%%%%%%%%%%%%%%%%%%%%%%%%%%%%%%%%%%%%%%%


\begin{document}

\name{Iago Lopes}
\tagline{Graduando em astronomia com forte interesse na área de ciência de dados. \\}
\photo[round]{_MG_7749.jpg}{\dimexpr \headerheight-\marginbottom}   % make photo exactly match the header with margintop/marginright/marginbottom as margin

\makeheader

\highlightbar{

    \section{Contato}
    
    \email{iagolops2012@gmail.com}
    \phone{(31) 99766-6595}
    \location{Rio de Janeiro, RJ}
    \vspace{0.0em}
    \github{@iagolops}{https://github.com/iagolops}
    \linkedin{Iago Lopes}{https://www.linkedin.com/in/iago-lops/}
    
    \section{Habilidades}
    
    \skillsection{Programação}
    \skill{Python}{3}
    \skill{Java}{1}
    \skill{LaTeX}{2}
    %\skill{SQL}{5}
    %\skill{Fortran}{3}
    
    \vspace{0.5em}
    \skillsection{Sistemas Operacionais}
    \skill{Linux}{2}
    \skill{Windows}{3}
    
    \vspace{0.5em}
    \skillsection{Software \& Ferramentas}
    \skill{Visualização}{2}
    (e.g. matplotlib, ...)\\
    \skill{Análise de dados}{2}
    (e.g. numpy, scipy, pandas, ...)\\
    \skill{Machine Learning}{1}
    (e.g. keras, ...)\\
    
    \vspace{0.5em}
    \skillsection{Conhecimentos adicionais}
    Estatística básica
    
    \vspace{0.5em}
    \skillsection{Linguagens}
    \skill{Inglês}{3}
    \bigskip
    
    \section{Certificados}
    \simpleskill{Coursera Supervised Machine Learning: Regression and Classification}
    \simpleskill{Coursera Advanced Learning Algorithms}

}
\mainbar{
    \section{Sobre mim}
        Sou um graduando em astronomia pela UFRJ com pesquisa em machine learning aplicado à estimativa de redshifts. Tenho interesse nas áreas computacionais e de análise de dados. Sou uma pessoa focada e minha prioridade é desenvolver meu conhecimento e habilidades.Procuro minha primeira oportunidade no mercado de ciência de dados.
    
    \section[\faGears]{Histórico de trabalho}
    \job{Jan 2023}
        {ORGUEL Vespasiano-MG}
        {Estagiário}
        {Trabalhava como projetista e fazia projetos para fôrmas e andaimes utilizando o AutoCAD}
    
    \section[\faMortarBoard]{Formação}
     %   \job{03/2026 - 12/2027}
    %   $ {University of Oxford - UK}
      %  {Mestrado em astronomia}
      %  {}
    
        \job{04/2022}
        {UFRJ Rio de Janeiro-RJ}
        {Graduação em astronomia}
        {}

        \job{04/2019 - 02/2022}
        {IFMG Santa Luzia-MG}
        {Técnico em edificações}
        {}

    \section{Experiências}
    %Pesquisa de graduação focada na estimativa de redshift analisando dados.\\
    IC em cosmologia com machine learning.\\
    IC concluída no Museu de Astronomia (MAST) sobre astronomia cultural.\\
    Jogador de vôlei na base do Minas Tênis Clube em 2017.\\
Curso de robótica e programação para crianças aos 11 anos de idade.\\
    
   % Pesquisa de mestrado focada em análise de dados usando o python.\\
    %Membro da corporação LSST.\\


    \vspace{0.5em}

    \section{Habilidades Gerais}
    \smallskip % additional skip because tag outlines use up space
    \tag{Trabalho em equipe}
    \tag{Liderança}
    \tag{Organização}
    \tag{Dedicação}
    \tag{Facilidade de aprendizado}
    
    \vspace{0.5em}
    
    \section{dia-a-dia}
    % This is taken from AltaCV
    % see https://github.com/liantze/AltaCV for details
    \wheelchart{1.5cm}{0.5cm}{% outer and inner diameter
        8/8em/accent!20/Dormindo,          % comma-separated list of
        9/8em/accent!90/Estudando,    % fraction of 24 / line length / color / label
        2/8em/accent!60/Treinando,          % here, the color is shades of the accent color
        2/8em/accent!50/Tempo livre,
        3/8em/accent/Estudando outras áreas
    }
}
\makebody
\clearpage


\pagestyle{highlightmain}

% The highlightbar needs to be filled to display mainbar contents correctly in singlesised mode
% For an empty highlightbar, fill with empty space
\highlightbar{\hfill}
\mainbar{

    \section{Another section}
    
    This page uses the page style \texttt{highlightmain} which shows the highlight bar (gray) and the main part (white background) but omits the header. 
    The default page style is \texttt{headerhighlightmain} with all three elements.
    If you don't want header, nor highlight bar, use page style \texttt{\textbackslash pagestyle\{empty\}}.
    \medskip
    Neither main, nor highlight bar must be filled to make this template work.
    It is possible to use a page style with the highlight bar but leave it empty by setting an empty highlightbar \texttt{\textbackslash highlightbar\{\}}.

    \vspace{0.5em}
    \subsection{Subsection 1}
    Demonstrate subsections.
    
    \subsection{Subsection 2}
    Subsection are also bold face but a smaller font then section. They also omit the rule.
    

}
\end{document}